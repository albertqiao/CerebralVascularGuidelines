\documentclass[UTF8]{ctexart}

\begin{document}

\title{笔记---中国急性缺血性脑卒中早期血管内介入诊疗指南2018}
\author{Qiao}
\date{\today}

\maketitle

\section{急性缺血性脑卒中早期血管内介入治疗}

	\subsection{适应证}
		\begin{enumerate}
			\item 年龄>18
			\item 大血管闭塞卒中患者尽早实施血管内介入治疗
			\begin{enumerate}
				\item 前循环闭塞<6h
				\item 前循环闭塞6-24h $\Rightarrow$ 严格影像学筛选
				\item 后循环闭塞<24h
			\end{enumerate}
			\item CT排除出血
			\item 影像学证实大血管闭塞 $\Rightarrow$ CTA
			\item 患者或法定代理人签署知情同意书
		\end{enumerate}
	
	\subsection{禁忌证}
		\begin{enumerate}
			\item 若进行动脉溶栓,参考静脉溶栓禁忌证标准
			\item 活动性出血或出血倾向
			\item 严重心、肝、肾功能不全
			\item 血糖 <2.7 mmol/L 或 >22.2 mmol/L
			\item 药物无法控制的严重高血压
		\end{enumerate}

\section{血管内机械取栓}

	\subsection{推荐意见}
		\begin{enumerate}
			\item 急性缺血性脑卒,满足下列条件,可采用血管内介入治疗
			\begin{enumerate}
				\item 发病前mRS为0或1分
				\item 明确病因为颈内动脉或M1闭塞
				\item 年龄 $\geq$ 18
				\item NIHSS $\geq$ 6
				\item ASPECTS $\geq$ 6
				\item 动脉穿刺时间控制在发病6h内
			\end{enumerate}
		\end{enumerate}


\end{document}